\documentclass{ajhlabreport}
\usepackage{ajhdsp}

\newcommand{\generatedfigw}[2]{
	\includegraphics[width=#1\textwidth]{generated/#2.png}
}

\setlength{\parindent}{0bp}

\datedue{December 6, 2012}
\class{EE4252}
\submittedto{\href{mailto:yliu18@mtu.edu}{Yang Liu}}
\pretitle{Lab 9 Report}
\title{IIR Filters}
\author{\href{mailto:ahirzel@mtu.edu}{Alex Hirzel}}

\begin{document}
\maketitle




\chapter{Project 1: Basic Routines for Digital Filter Design}%%%%%%%%%%%%%%%%%%%



\section{Exercise 1: IIR Filter Design}%%%%%%%%%%%%%%%%%%%%%%%%%%%%%%%%%%%%%%%%%

\begin{figure}[H]
\centering
\generatedfigw{0.8}{../p1e1}
\caption{Hard copy of \texttt{dfdiir} showing the design of the desired filter.}
\end{figure}

\begin{figure}[H]
\centering
\begin{tikzpicture}
\begin{axis}[continuous fr, height=3in, width=5in,
% http://tex.stackexchange.com/questions/36442/aligning-subplots-in-a-pgfplots-figure
yticklabel style={text width=2.35em,align=right},
% http://tex.stackexchange.com/questions/51095/reduce-font-size-and-keep-label-position-in-bar-chart-using-pgf-tikz
tick label style={font=\footnotesize},
xlabel={$f$ ($\hertz$)},
xtick={1000,2000,3000,4000,5000},
xmax=5100,
ylabel=$H(f)$,
ymax=1.1,
]
\addplot table[col sep=comma,x index=0, y index=1] {generated/p1e1.csv};
\end{axis}
\end{tikzpicture}
\caption{Plot of $\abs{H(f)}$ versus frequency in $\hertz$.}
\end{figure}



\newpage
\section{Exercise 2: s2z mappings}%%%%%%%%%%%%%%%%%%%%%%%%%%%%%%%%%%%%%%%%%%%%%%

\begin{enumerate}[(a)]
%
\item The results of converting $H(s) = 1/(s + 2)$ are:
\begin{align*}
H_0(z) = \frac{0.1967 z}{z - 0.6065}
\qquad
H_1(z) = \frac{0.1967 z}{z - 0.6065}
\qquad
H_2(z) = \frac{0.1 - 0.1 z}{z - 0.6}
\end{align*}
\item All filters are overplotted below.
\begin{figure}[H]\centering\generatedfigw{0.9}{p1e2-magnitudes}\end{figure}
%
\item It can be seen from above that $H_1(z)$ from above most closely matches
$H(s)$. This is because the red and blue lines are closer to one another than
the blue and teal lines.
%
\item Shown below are their impulse responses $h_{1\{0,1,2\}}[n]$.
\begin{align*}
H_{10}(z) &= \frac{0.1967 z}{z - 0.6065}
&
h_{10}[n] &= 0.19673 (0.6065)^n u[n]
\\
H_{11}(z) &= \frac{0.9837 z + 0.9837}{z - 0.6065}
&
h_{11}[n] &= 0.26056 (0.6065)^n u[n] - 0.16219 \delta[n]
\\
H_{12}(z) &= \frac{0.1 - 0.1 z}{z - 0.6}
&
h_{12}[n] &= 0.32438 (0.6065)^n u[n] - 0.32438 \delta[n]
\end{align*}
\begin{figure}[H]
\begin{tikzpicture}
\begin{axis}[discrete ir,
xmin=-0.00001, ymax=0.4,
width=\textwidth, height=2.5in]
\addplot { 0.19673 * (0.6065^x) };
\addplot { 0.26056 * (0.6065^x) - 0.16219 * floor(1/(1+x)) };
\addplot { 0.32438 * (0.6065^x) - 0.32438 * floor(1/(1+x)) };
\end{axis}
\end{tikzpicture}
\end{figure}
%
\end{enumerate}




\newpage
\chapter{Project 2: GUI-Based Digital Filter Design}%%%%%%%%%%%%%%%%%%%%%%%%%%%%

The impulse invariant and bilinear transform Butterworth designs produced by
\texttt{diirgui} are:

\begin{align*}
H_\text{II}(z) &= \frac{0.0005238 z^{10} + 0.02412 z^8 + 0.09109 z^6 + 0.01531 z^4 - 0.006237 z^2 - 0.001179}{z^{10} + 2.811 z^8 + 3.432 z^6 + 2.216 z^4 + 0.746 z^2 + 0.1038} \\
H_\text{Bi}(z) &= \frac{0.03047 z^6 - 0.0914 z^4 + 0.0914 z^2 - 0.03047}{z^6 + 1.483 z^4 + 0.9296 z^2 + 0.2033}
\end{align*}

Each filter design meets specs. The two yield filters that are \emph{not} the
same order. The filter designed via the bilinear transform is more economical
with order six rather than ten.

\begin{figure}[H]
\centering
\begin{tikzpicture}
\begin{axis}[height=2.5in, width=5in,
grid,no marks,
axis x line = bottom,
xlabel = {Digital frequency, $F$},
xtick = {0.1,0.2,0.3,0.4,0.5},
xmin = 0, xmax = 0.52,
axis y line = left,
ylabel = {Filter attenuation, $20\log{\abs{H(z)}}$ ($\deci\bel$)},
ymin = -50, ymax = 2,
]
\addplot table[col sep=comma,x index=0, y index=1] {generated/p2-butt.csv};
\addplot table[col sep=comma,x index=0, y index=2] {generated/p2-butt.csv};
\node[coordinate,pin=above left:{\color{red}$-30.28\deci\bel$}] at (axis cs:0.1,-30.2770) {};
\node[coordinate,pin=below right:{\color{blue}$-31.75\deci\bel$}] at (axis cs:0.1,-31.7521) {};
\legend{$H_\text{Bi}(z)$, $H_\text{II}(z)$}
\end{axis}
\end{tikzpicture}
\end{figure}

The Chebyshev filters also have different orders (eight and six, respectively,
for the impulse invariant and bilinear transform designs). Both designs still
meet specs. The filter designed via the bilinear transform is more economical
with order six rather than eight.

\begin{align*}
H_\text{II}(z) &= \frac{-z^8 - 3.067 z^6 - 3.837 z^4 - 2.29 z^2 - 0.5474}{0.001235 z^8 - 0.005121 z^6 - 0.04221 z^4 - 0.0116 z^2 - 0.0003102} \\
H_\text{Bi}(z) &= \frac{z^6 + 2.138 z^4 + 1.769 z^2 + 0.5398}{0.01147 z^6 - 0.03442 z^4 + 0.03442 z^2 - 0.01147}
\end{align*}

\begin{figure}[H]
\centering
\begin{tikzpicture}
\begin{axis}[height=2.5in, width=5in,
grid,no marks,
axis x line = bottom,
xlabel = {Digital frequency, $F$},
xtick = {0.1,0.2,0.3,0.4,0.5},
xmin = 0, xmax = 0.52,
axis y line = left,
ylabel = {Filter attenuation, $20\log{\abs{H(z)}}$ ($\deci\bel$)},
ymin = -50, ymax = 2,
]
\addplot table[col sep=comma,x index=0, y index=1] {generated/p2-cheb.csv};
\addplot table[col sep=comma,x index=0, y index=2] {generated/p2-cheb.csv};
\node[coordinate,pin=above left:{\color{red}$-39.00\deci\bel$}] at (axis cs:0.1,-39.0450) {};
\node[coordinate,pin=right:{\color{blue}$-41.42\deci\bel$}] at (axis cs:0.1,-43.4200) {};
\legend{$H_\text{Bi}(z)$, $H_\text{II}(z)$}
\end{axis}
\end{tikzpicture}
\end{figure}




\chapter{Project 3: A Digital Audio Equalizer}%%%%%%%%%%%%%%%%%%%%%%%%%%%%%%%%%%

The response varies as an equalizer would be expected to work: there are five
controllable peaks corresponding to the \texttt{g} input to \texttt{audioeq},
and the flattest response was found using \texttt{audioeq(0.5, [1 1 1 0.5
2.1])}. The passband ripple is on the order of $1\deci\bel$.

\begin{figure}[H]
\centering
\begin{tikzpicture}
\begin{axis}[height=2.5in, width=5in,
grid,no marks,
axis x line = bottom,
xlabel = {Digital frequency, $F$},
xtick = {0.1,0.2,0.3,0.4,0.5},
xmin = 0, xmax = 0.52,
axis y line = left,
ylabel = {Filter attenuation, $20\log{\abs{H(z)}}$ ($\deci\bel$)},
ymin = -10, ymax = 10,
]
\addplot table[col sep=comma,x index=0, y index=1] {generated/p3-flattest.csv};
\end{axis}
\end{tikzpicture}
\end{figure}




\chapter{Project 4: Stability of IIR Filters}%%%%%%%%%%%%%%%%%%%%%%%%%%%%%%%%%%%

\begin{enumerate}[(a)]
%
\item The result is approximately
\begin{align*}
H(z) \approx \frac{10^{-7} (1.103 z^5 + 1.125 z^4 - 2.163 z^3 - 2.207 z^2 + 1.06 z + 1.082)}{z^5 - 4.736 z^4 + 8.969 z^3 - 8.488 z^2 + 4.015 z - 0.7594}\text.
\end{align*}
%
\setcounter{enumi}{2}
\item The bode plot is shown below. The truncated version is shown in
{\color{red}red}.
\begin{figure}[H]\centering\generatedfigw{0.9}{p4}\end{figure}
%
\item The numerator of both $H(z)$ and $H_T(z)$ have zeroes that are on the unit
circle (magnitude 1.000002633680229). For the purposes of this analysis, then,
the filter will be minimum-phase if denominator poles are inside the unit
circle. In the case of $H(z)$, all poles are inside the unit circle, so $H(z)$
is stable and minimum-phase.
%
\item $H_T(z)$ has the same numerator as $H(z)$, but the denominator of $H_T(z)$
has a pole that is outside the unit circle. Therefore, $H_T(z)$ is not stable
nor is it minimum-phase.
%
\item Truncation to 9 significant figures results in a stable filter (which is
therefore minimum-phase). Expectedly, truncation to 5 significant figures
results in a filter that is even less stable than $H_T(z)$ (i.e. pole radius is
even larger), a non-minimum-phase result.
%
\end{enumerate}


\newpage
\chapter{Appendix: MATLAB Source Code}

What follows is a listing of the MATLAB source code (\autoref{lst:source-code})
and the text output of this code (\autoref{lst:diary-txt}) used to generate the
figures and other information presented in this report.

\lstinputlisting[caption={The MATLAB script used for this report, \texttt{Lab09\_ahirzel.m}.},label={lst:source-code},style=MATLABcode]{Lab09_ahirzel.m}
\lstinputlisting[caption={The output of listing \ref{lst:source-code}, \texttt{diary.txt}.},label={lst:diary-txt}]{generated/diary.txt}

\end{document}
