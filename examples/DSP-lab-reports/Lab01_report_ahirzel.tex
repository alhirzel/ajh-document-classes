\documentclass{ajhlabreport}

\newcommand{\savedfig}[2]{
	\begin{figure}[H]
	\centering
	\includegraphics[width=0.65\textwidth]{#1}
	\caption{#2}
	\end{figure}
}

\newcommand{\savedhardcopy}[2]{
	\begin{figure}[H]
	\centering
	\includegraphics[trim=20 180 138 275,clip,width=0.8\textwidth]{#1}
	\caption{#2}
	\end{figure}
}


\datedue{September 27, 2012}
\class{EE4252}
\submittedto{Yang Liu}
\pretitle{Lab 1 Report}
\title{Introduction to MATLAB}
\author{\href{mailto:ahirzel@mtu.edu}{Alex Hirzel}}

\begin{document}

% Assignment sheet on the front
\includepdf[pages={1,2,3}]{lab1.pdf}
\cleardoublepage

\maketitle

\chapter{Introduction}

This report presents the information from Lab 1, which consists of the analysis
of the following three systems:
\[
H(s) = \frac{1}{s+4}
\qquad\qquad
y[n]-0.5y[n-1]=2x[n]
\qquad\qquad
y[n]-1.4y[n-1]+C y[n-2]=x[n]
\]

\chapter{Project 1}

\savedfig{p1_step_impulse.png}{Step response and impulse response of $1/(s+4)$ as generated for question 1(b).\label{fig:step-impulse}}
\savedfig{p1_frequency.png}{Frequency response of $1/(s+4)$ as generated for question 1(c).\label{fig:frequency}}
\savedhardcopy{p1_bode.pdf}{Bode plot of $1/(s+4)$ as captured for question 1(d).\label{fig:bode}}
\savedfig{p1_rise_time.png}{Plot of $1/(s+4)$ as generated from \texttt{trbw} for question 1(e). The rise-time--bandwidth product is $0.64\hertz \cdot 0.54875\second = \boxed{0.3512}$.\label{fig:rise-time}}

\chapter{Project 2}

Two systems are analyzed in this project:
\begin{align}
y[n]-0.5y[n-1]=2x[n] \label{eq:2-first}\\
y[n]-1.4y[n-1]+C y[n-2]=x[n] \label{eq:2-second}
\end{align}

\savedfig{p2_1_responses.png}{The step response and response to $0.4^n u(t)$ of system \eqref{eq:2-first} as generated by \texttt{dtsimgui} for question 1(a).\label{fig:21-dtsimgui-responses}}
\savedfig{p2_1_filter_response.png}{The step response of system \eqref{eq:2-first} as generated by \texttt{filter} for question 1(b).\label{fig:21-filter-response}}

Shown in figures \ref{fig:21-dtsimgui-responses} and
\ref{fig:21-filter-response} (respectively) are the responses of system
\eqref{eq:2-first} to a step input. These two figures are in exact agreement on
the step response, which is to be expected.

\savedhardcopy{p22response5.pdf}{Hard copy of \texttt{dtsimgui} GUI for $C=0.5$.\label{fig:p22response5}}
\savedhardcopy{p22response7.pdf}{Hard copy of \texttt{dtsimgui} GUI for $C=0.7$.\label{fig:p22response7}}
\savedfig{p2_1_final_values.png}{Step responses of system \eqref{eq:2-second} as generated for question 2(a). Shown are step responses---from top to bottom---for $C=0.5$ ({\color{red}red}), $C=0.6$ ({\color{green}green}), $C=0.7$ ({\color{blue}blue}) and $C=0.8$ (black).\label{fig:final-values}}

Shown in many preceeding figures are the responses of system \eqref{eq:2-second}
to a step input. \autoref{tab:final-value-vs-C} shows the final value taken on
by the system as a function of the constant $C$. This varying coefficient allows
the exploration of the stability of the filter because for some values of $C$
the filter is both stable and divergent.

\begin{table}[H]
\centering
\caption{Final value of the step response of $y[n] - 1.4y[n-1] + C y[n-2] =
x[n]$ for varying $C$.\label{tab:final-value-vs-C}}
\begin{tabular}{cc}
$C$ & Final value \\
\midrule
0.5 & 10 \\
0.6 & 5 \\
0.7 & 3.3334 \\
0.8 & 2.5027 \\
\end{tabular}
\end{table}

The characteristic equation $z^2 - 1.4z + C = 0$ of system \eqref{eq:2-second}
has roots according to the quadratic equation\footnote{$ax^2+bx+c=0 \iff x=(-b
\pm \sqrt{b^2 - 4 a c})/(2a)$.}. These roots are less than unity when \[ \left|
\frac{1.4 \pm \sqrt{1.96 - 4C}}{2} \right| < 1 \quad \Rightarrow \quad \boxed{C
> 0.4 \text{ or } C > 0.4}\text{.} \] It is under the opposite of this condition
that the system is unstable. That is to say: when $C<0.4$, the system is
unstable and when $C=0.4$ the system is marginally stable. This stability
criterion is rooted in the iterative nature of the filter. Specifically, looking
at the difference equation form of the system, \[ y[n] = x[n] - 1.4y[n-1] + C
y[n-2] \] it can be inferred that if approximately steady-state conditions are
assumed---specifically that $y[n]$ is not changing much during each iteration
(making $y[n-1]\approx{}y[n-2]$) and $x=1$ (because a step response is under
consideration)---then the final value of $y[n]$ is approximated by:
\begin{align}
\smash{\stackrel{\approx}{y}}[n] &= 1 - 1.4y[n-1] + C y[n-1] \notag \\
                                 &= 1 - (1.4 - C) y[n-1] \label{eq:approx}
\end{align}
which will grow without bound unless $1.4 - C < 1$ (i.e. $C>0.4$) due to the
recursive nature of the filter. As can be seen in \eqref{eq:approx}, if $C=0.4$,
then $y[n]=1 - y[n-1]$. This implies marignal stability about $y[n]=1$ with
$C=0.4$.

\begin{figure}[H]
\centering
\includegraphics[width=\textwidth]{p2_1_unstables.png}
\caption{Step responses of system \eqref{eq:2-second} as generated for question 2(b). Shown are step responses---from top to bottom---for $C=0.4$ ({\color{red}red}), $C=0.3$ ({\color{green}green}) and $C=0.2$ ({\color{blue}blue}).\label{fig:unstables}}
\end{figure}

The same story is told by \autoref{fig:unstables}, which shows the step
response with $C=0.4,0.3,0.2$. The left-most plot ($C=0.4$) shows marginal
stability in the form of a divergent, perfectly linear output. Decreasing
values of $C$ force more rapid divergence.

\newpage
\appendix
\chapter{Appendix: MATLAB Source Code}

What follows is a listing of the MATLAB source code (listing
\ref{lst:lab01-ahirzel-m})---and the output of this code (listing
\ref{lst:diary-txt})---used to generate the figures and other information
presented in this report.

\lstinputlisting[caption={The MATLAB script used for this report, \texttt{Lab01\_ahirzel.m}.},label={lst:lab01-ahirzel-m},style=MATLABcode]{Lab01_ahirzel.m}
\lstinputlisting[caption={The output of listing \ref{lst:lab01-ahirzel-m}, \texttt{diary.txt}.},label={lst:diary-txt}]{diary.txt}

\end{document}
