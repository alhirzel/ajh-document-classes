\documentclass[dsp]{ajhlabreport}
\usepackage{multirow,booktabs}

\newcommand{\generatedfigw}[2]{
	\includegraphics[width=#1\textwidth]{generated/#2.png}
}

\providecommand{\n}[0]{[n]}

\datedue{October 11, 2012}
\class{EE4252}
\submittedto{Yang Liu}
\pretitle{Lab 3 Report}
\title{Discrete Convolution}
\author{Alex Hirzel}

\begin{document}

\maketitle



\chapter{Project 1}%%%%%%%%%%%%%%%%%%%%%%%%%%%%%%%%%%%%%%%%%%%%%%%%%%%%%%%%%%%%%

\begin{figure}[H]
\centering
\generatedfigw{1}{p1_inputs}
\caption{$x[n]$ (left) and $h[n]$ (right) used for convolution in Project 1.\label{fig:p1-inputs}}
\end{figure}

\begin{figure}[p]
\centering
\subbottom[The convolution result $y_1\n = x\n \conv h\n$.\label{fig:p1-y1}]{\generatedfigw{1}{p1_y1}}
\subbottom[The convolution result $y_2\n = h\n \conv x\n$.\label{fig:p1-y2}]{\generatedfigw{1}{p1_y2}}
\subbottom[The convolution result $y_3\n = x\n \conv x\n$.\label{fig:p1-y3}]{\generatedfigw{1}{p1_y3}}
\subbottom[The convolution result $y_4\n = h\n \conv h\n$.\label{fig:p1-y4}]{\generatedfigw{1}{p1_y4}}
\caption{The outputs from convolution in Project 1.\label{fig:p1}}
\end{figure}

The properties of convolution are verified using \autoref{fig:p1-inputs} as
follows:
%
\begin{itemize}
%
\item{The \textbf{order property is confirmed} because \autoref{fig:p1-y1} and
\autoref{fig:p1-y2} are identical;}
%
\item{The \textbf{length property is verified} by noting that
\autoref{fig:p1-y1} and \autoref{fig:p1-y2} are equal in length;}
%
\item{The \textbf{sum property is verified} using \autoref{lst:diary-txt} by
noting that $\sum x[n]=36$ and $\sum h[x]=0$ and that the lengths of $y_1 \cdots
y_4$ are in agreement with the sum property (i.e. all are zero except $\sum y_3
= 36^2 = 1296$).}
%
\end{itemize}
%
Additionally, symmetry of the outputs is observed as shown in the below table.
%
\begin{table}[H]
\centering
\begin{tabular}{r@{\ \ensuremath{\conv}\ }l@{\hspace{0.5in}}r@{\hspace{0.1in}i.e. }l}
\multicolumn{2}{c@{\hspace{0.5in}}}{Input} & \multicolumn{2}{@{}c}{Output} \\
\midrule
even & odd  & even & $y_1$, $y_2$ \\
odd  & odd  & odd  & $y_3$ \\
even & even & even & $y_4$
\end{tabular}
\end{table}



\newpage
\chapter{Project 2}%%%%%%%%%%%%%%%%%%%%%%%%%%%%%%%%%%%%%%%%%%%%%%%%%%%%%%%%%%%%%

The input and response ($x[n]$ and $y[n]$, respectively) are shown in
\autoref{fig:p2}. \textbf{The period of both $x[n]$ and $y[n]$ is 10.} This
makes sense because $h[n]$ is non-periodic and is an LTI system; this means a
periodic input will result in an output that is periodic with the same
frequency.

\begin{figure}[H]
\centering
\generatedfigw{1}{p2}
\caption{The input $x[n]$ and response $y[n]$ from Project 2.\label{fig:p2}}
\end{figure}



\newpage
\chapter{Project 3}%%%%%%%%%%%%%%%%%%%%%%%%%%%%%%%%%%%%%%%%%%%%%%%%%%%%%%%%%%%%%

Shown in \autoref{fig:p3} are the results of convolution as plotted from
\autoref{lst:diary-txt}. Note that it is expected for $y_n[n] = y[n]$, which is
seen to be true.

\pgfplotsset{discrete ir/.append style={ytick={0,40,...,200}}}

\pgfplotstableread{
1
5
14
30
51
75
100
117
117
98
58
27
7
}\dataone

\pgfplotstableread{
118
122
112
88
78
82
100
}\datatwo

\begin{figure}[H]
\centering
\providecommand{\n}[0]{[n]}
%
\subbottom[The convolution result $y_n\n = x\n \conv h\n$.\label{fig:p3-y1}]{
\begin{tikzpicture}
\begin{axis}[discrete ir, ylabel=$y_n\n$, xmax=12.5, ymax=159]
\addplot table[x expr=\coordindex, y index=0] {\dataone};
\end{axis}
\end{tikzpicture}
}
%
\subbottom[The convolution result $y_p\n = x\n \convp h\n$.\label{fig:p3-y2}]{
\begin{tikzpicture}
\begin{axis}[discrete ir, ylabel=$y_p\n$, xmax=12.5, ymax=159]
\addplot table[x expr=\coordindex, y index=0] {\datatwo};
\end{axis}
\end{tikzpicture}
}
%
\subbottom[The periodic convolution of the padded versions of $x\n$ and $h\n$ is
identical to $y_n$.\label{fig:p3-y3}]{
\begin{tikzpicture}
\begin{axis}[discrete ir, ylabel=$y\n$, xmax=12.5, ymax=159]
\addplot table[x expr=\coordindex, y index=0] {\dataone};
\end{axis}
\end{tikzpicture}
}
%
\caption{The outputs from convolution in Project 3.\label{fig:p3}}
\end{figure}



\newpage
\chapter{Project 4}%%%%%%%%%%%%%%%%%%%%%%%%%%%%%%%%%%%%%%%%%%%%%%%%%%%%%%%%%%%%%

(The hard copy from \texttt{chirpgui} is shown in \autoref{fig:p4-gui} in the
appendices.)

\begin{itemize}
\item{\autoref{fig:p4-y1} shows a bandpass filter with cutoff frequencies at
approximately $F=0.05$ and $F=0.15$.}
\item{\autoref{fig:p4-y2} shows a bandpass filter with an exponential drop-off
(starting at approximately $F=0.05$ and trailing off at $F=0.4$.}
\end{itemize}

The above filter types were rendered by looking to the power-spectral density
(plotted in the bottom left of each quad) and comparing against known filter
types (such as high-pass, low-pass, notch, etc.).

\begin{figure}[H]
\centering
\generatedfigw{1}{p4_linear_freq}
\caption{Output signal versus time plot showing that frequency increases with time.\label{fig:p4-linear-freq}}
\end{figure}

\newpage

\begin{figure}[H]
\centering
\subbottom[$h_1\n = \delta\n - 0.4\cos{(0.4 n \pi)} \sinc{(0.2n)}$.\label{fig:p4-y1}]{\generatedfigw{1}{p4_y1}}
\subbottom[$h_2\n = 0.3 \sinc{(0.3 n)}$.\label{fig:p4-y2}]{\generatedfigw{1}{p4_y2}}
\caption{Spectral information for the two systems under consideration in Project 4.\label{fig:p4-y}}
\end{figure}



\newpage
\appendix

\chapter{Appendix: Miscellaneous figures}

\begin{figure}[H]
\centering
\generatedfigw{1}{../p4_gui}
\caption{Hard copy of \texttt{chirpgui} for Project 4.\label{fig:p4-gui}}
\end{figure}


\chapter{Appendix: MATLAB Source Code}

What follows is a listing of the MATLAB source code (listing
\ref{lst:source-code})---and the output of this code (listing
\ref{lst:diary-txt})---used to generate the figures and other information
presented in this report.

\lstinputlisting[caption={The MATLAB script used for this report, \texttt{Lab03\_ahirzel.m}.},label={lst:source-code},style=MATLABcode]{Lab03_ahirzel.m}
\lstinputlisting[caption={The output of listing \ref{lst:source-code}, \texttt{diary.txt}.},label={lst:diary-txt}]{generated/diary.txt}

\end{document}
