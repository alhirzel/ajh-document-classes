\documentclass{ajhlabreport}
\usepackage{ajhdsp}
\usepackage{verbatim}
\usetikzlibrary{pgfplots.groupplots}

\newcommand{\generatedfigw}[2]{
	\includegraphics[width=#1\textwidth]{generated/#2.png}
}

\providecommand{\n}[0]{[n]}

\datedue{November 1, 2012}
\class{EE4252}
\submittedto{\href{mailto:yliu18@mtu.edu}{Yang Liu}}
\pretitle{Lab 6 Report}
\title{The DFT and FFT}
\author{\href{mailto:ahirzel@mtu.edu}{Alex Hirzel}}

\begin{document}
\maketitle



\chapter{Project 1: Sampling a Combination of Sinusoids}

This project investigates the sampling of the following signal at multiple
rates.
%
\begin{align}
x(t) = \sin{(200 \pi t)} + \sin{(480 \pi t)}
\label{eq:p1-system}
\end{align}
%
MATLAB was used to determine whether aliasing occurs at the indicated sample
rate. MATLAB was also used to calculate the minimum number of samples required
to eliminate leakage ($N_\text{min}$). This data is shown below.
%
\newcommand{\B}[1]{{\color{blue}#1}}
\newcommand{\R}[1]{{\color{red}#1}}
\newcommand{\X}[2]{${#1}^\text{fig \ref{fig:p1-#2}}$}
%
\begin{table}[H]
\centering
\begin{tabular}{ccccc}
%
$S$          & Aliasing? & $N_\text{min}$     & $3N_\text{min}$    & $3N_\text{min} + 1$   \\
\midrule
$1700\hertz$ & no        & \X{85}{1700-85}    & \X{255}{1700-255}  & \X{\B{263}}{1700-263} \\
$360\hertz$  & yes       & \X{\R{18}}{360-18} & \X{\R{54}}{360-54} & \X{\R{62}}{360-62}    \\
$120\hertz$  & yes       & \X{\R{6}}{120-6}   & \X{\R{18}}{120-18} & \X{\R{26}}{120-26}    \\
$50\hertz$   & yes       & \X{\R{5}}{50-5}    & \X{\R{15}}{50-15}  & \X{\R{23}}{50-23}
%
\end{tabular}
\end{table}
%
\noindent{}Numbers typeset in color indicate combinations of sampling rate and
DFT size that will not result in identical reconstruction (i.e. $x(t) \ne
x_\text{r}(t)$ due to \R{aliasing if red} or \B{leakage if blue}). Required
plots are shown in figures \ref{fig:p1-1700}, \ref{fig:p1-360}, \ref{fig:p1-120}
and \ref{fig:p1-50}. Questions (c), (f) and (g) from the lab hand-out are
addressed in the figure captions.

\newcommand{\irplotfile}[2]{
\pgfplotstableread{generated/#1.txt}{\datafromfilefragile}
\centering
\begin{tikzpicture}
\begin{axis}[discrete ir, #2, height=1.5in, width=\textwidth,
% http://tex.stackexchange.com/questions/36442/aligning-subplots-in-a-pgfplots-figure
yticklabel style={text width=2.35em,align=right},
% http://tex.stackexchange.com/questions/51095/reduce-font-size-and-keep-label-position-in-bar-chart-using-pgf-tikz
tick label style={font=\footnotesize},
]
\addplot table[x expr=\coordindex, y index=0] \datafromfilefragile;
\end{axis}
\end{tikzpicture}
}

\begin{figure}[H]\centering
% 1700 Hz   n=85
\subbottom[The $85$-point DFT of $x(t)$ sampled at $1700\hertz$ shows peaks at
analog frequencies of $\pm100\hertz$ and $\pm240\hertz$ as expected. This is
because the sampling frequency is high enough and the DFT window captures an
integral number of periods of the underlying sinusoids.\label{fig:p1-1700-85}
]{\irplotfile{dft1700-min}{ylabel=$X_\text{DFT}\n$,xtick={5, 12, 73, 80},
ytick={0, 42.5}, xmax=87, ymax=50}}
% 1700 Hz   n=255
\subbottom[The $255$-point DFT of $x(t)$ sampled at $1700\hertz$. The commentary
from \autoref{fig:p1-1700-85} (above) applies to this case as
well.\label{fig:p1-1700-255}
]{\irplotfile{dft1700-3min}{ylabel=$X_\text{DFT}\n$,xtick={15, 36, 219,
240}, ytick={0, 127.5}, xmax=261, ymax=150}}
% 1700 Hz   n=263
\subbottom[The $263$-point DFT of $x(t)$ sampled at $1700\hertz$ shows peaks at
many analog frequencies near $\pm100\hertz$ and $\pm240\hertz$. This is expected
due to leakage.\label{fig:p1-1700-263}
]{\irplotfile{dft1700-3min8}{ylabel=$X_\text{DFT}\n$,xtick={15.462, 37.109,
225.738, 247.385}, ytick={0, 89.834, 126.93}, xmax=269, ymax=150}}
%
\caption{Sampling system \eqref{eq:p1-system} at $1700\hertz$.\label{fig:p1-1700}}
\end{figure}

\begin{figure}[H]\centering
% 360 Hz   n=18
\subbottom[The $18$-point DFT of $x(t)$ sampled at $360\hertz$ shows peaks at
$\pm100\hertz$ and $\mp120\hertz$. These are as expected because the $240\hertz$
frequency component in the original signal is aliased and phase-reversed
(because $240\hertz - 360\hertz = -120\hertz$).\label{fig:p1-360-18}
]{\irplotfile{dft360-min}{ylabel=$X_\text{DFT}\n$,xtick={5, 6, 12, 13},
ytick={0, 9}, xmax=18.25, ymax=12}}
% 360 Hz   n=54
\subbottom[The $54$-point DFT of $x(t)$ sampled at $360\hertz$. The commentary
from \autoref{fig:p1-360-18} (above) applies to this case as
well.\label{fig:p1-360-54}
]{\irplotfile{dft360-3min}{ylabel=$X_\text{DFT}\n$,xtick={15, 18, 36, 39},
ytick={0, 27}, xmax=54.75, ymax=35}}
% 360 Hz   n=62
\subbottom[The $62$-point DFT of $x(t)$ sampled at $360\hertz$ shows peaks at
many analog frequencies near $\pm100\hertz$ and $\mp120\hertz$. This is expected
due to leakage.\label{fig:p1-360-62}
]{\irplotfile{dft360-3min8}{ylabel=$X_\text{DFT}\n$,xtick={17.222, 20.666,
41.333, 44.777}, ytick={0, 14.207, 24.627, 27.858}, xmax=62.75, ymax=35}}
%
\caption{Sampling system \eqref{eq:p1-system} at $360\hertz$.\label{fig:p1-360}}
\end{figure}

\begin{figure}[H]\centering
% 120 Hz   n=6
\subbottom[The $6$-point DFT of $x(t)$ sampled at $120\hertz$ shows peaks at
$\mp20\hertz$. These are as expected because the $240\hertz$ frequency component
in the original signal is aliased to $0\hertz$ (with the DC value cancling out
due to the even number of samples) and the $100\hertz$ component is aliased and
phase-reversed to $-20\hertz$ (because $100\hertz - 120\hertz =
-20\hertz$).\label{fig:p1-120-6}
]{\irplotfile{dft120-min}{ylabel=$X_\text{DFT}\n$,xtick={1, 5}, ytick={0, 3},
xmax=6.25, ymax=4}}
% 120 Hz   n=18
\subbottom[The $18$-point DFT of $x(t)$ sampled at $120\hertz$. The commentary
from \autoref{fig:p1-120-6} (above) applies to this case as
well.\label{fig:p1-120-18}
]{\irplotfile{dft120-3min}{ylabel=$X_\text{DFT}\n$,xtick={3, 15}, ytick={0, 9},
xmax=18.75, ymax=12}}
% 120 Hz   n=26
\subbottom[The $26$-point DFT of $x(t)$ sampled at $120\hertz$ shows peaks at
many analog frequencies near $\mp20\hertz$. This is expected due to
leakage.\label{fig:p1-120-26}
]{\irplotfile{dft120-3min8}{ylabel=$X_\text{DFT}\n$,xtick={4.333, 21.666},
ytick={0, 10.507, 5.6362}, xmax=26.75, ymax=15}}
%
\caption{Sampling system \eqref{eq:p1-system} at $120\hertz$.\label{fig:p1-120}}
\end{figure}

\begin{figure}[H]\centering
% 50 Hz   n=5
\subbottom[The $5$-point DFT of $x(t)$ sampled at $50\hertz$ shows peaks at
$\mp10\hertz$. These are as expected because the $100\hertz$ frequency component
in the original signal is aliased to $0\hertz$ (with the DC value canceling out
due to the even number of samples) and the $240\hertz$ component is aliased and
phase-reversed to $-10\hertz$ (because $240\hertz - 5\cdot50\hertz =
-10\hertz$).\label{fig:p1-50-5}
]{\irplotfile{dft50-min}{ylabel=$X_\text{DFT}\n$,xtick={1, 4}, ytick={0, 2.5},
xmax=5.25, ymax=3}}
% 50 Hz   n=15
\subbottom[The $5$-point DFT of $x(t)$ sampled at $50\hertz$. The commentary
from \autoref{fig:p1-50-5} (above) applies to this case as
well.\label{fig:p1-50-15}
]{\irplotfile{dft50-3min}{ylabel=$X_\text{DFT}\n$,xtick={3, 12}, ytick={0, 7.5},
xmax=15.75, ymax=10}}
% 50 Hz   n=23
\subbottom[The $23$-point DFT of $x(t)$ sampled at $50\hertz$ shows peaks at
many analog frequencies near $\mp10\hertz$. This is expected due to
leakage.\label{fig:p1-50-23}
]{\irplotfile{dft50-3min8}{ylabel=$X_\text{DFT}\n$,xtick={4.6, 18.4}, ytick={0,
6.2332, 8.3148}, xmax=23.75, ymax=10}}
%
\caption{Sampling system \eqref{eq:p1-system} at $50\hertz$.\label{fig:p1-50}}
\end{figure}



\chapter{Project 2: Leakage ($200\hertz$)}


\generatedfigw{1}{p2-200hz}
\renewcommand{\labelenumi}{(\alph{enumi})}
\begin{enumerate}
\item Frequency components at $10\hertz$ and $60\hertz$ are expected. Aliasing
does not occur in any plot because $f_s > 2 f_\text{max}$ (i.e. $200\hertz >
2\cdot60\hertz$).
\item No leakage occurs in plot $D_2$ because there is no "skirt" around
frequency components in this plot.
\item The remaining plots show leakage (i.e. $D_1$, $D_3$ and $D_4$). This is
concluded because there are no sharp peaks at $10\hertz$ or $60\hertz$ in these
plots.
\item Plot $D_1$ has the longest duration of any of the plots mentioned in (c);
this is because it is the most "sharp".
\end{enumerate}
\newpage



\chapter{Project 2: Leakage ($100\hertz$)}

\begin{figure}[H]
\centering
\generatedfigw{1}{p2-100hz}
\end{figure}
%
\renewcommand{\labelenumi}{(\alph{enumi})}
\begin{enumerate}
%
\item Frequency components at $10\hertz$ and $60\hertz$ are expected. Aliasing
is visible in all plots because the $60\hertz$ frequency component is aliased to
$40\hertz$ and phase-reversed. (This is because $60\hertz - 100\hertz =
-40\hertz$.)
%
\item No leakage occurs in plot $D_2$ because there is no "skirt" around
frequency components in this plot.
%
\item The remaining plots show leakage (i.e. $D_1$, $D_3$ and $D_4$). This is
concluded because there are no sharp peaks at $10\hertz$ or $40\hertz$ in these
plots.
%
\item Plot $D_1$ has the longest duration of any of the plots mentioned in (c);
this is because it is the most "sharp".
%
\end{enumerate}
\newpage



\chapter{Project 3: Sampling a Chirp Signal}

\renewcommand{\labelenumi}{(\alph{enumi})}
\begin{enumerate}
%
\item The highest frequency present is $2048 \cdot 2 = 4096\hertz$, which will
not be aliased by sampling at $8192\hertz$. The frequency seems to increase with
time in a smooth way, eventually fading due to the nonlinearity of my human
hearing.
%
\item The highest frequency present is $8192 \cdot 2 = 16384\hertz$, which will
most certainly be aliased by sampling at $8192\hertz$. The frequency increases
and decreases twice in $2\second$, which agrees with the above plot.
%
\item The below plot of \texttt{fa1} shows the frequency of the sound increases with
time. My ears agree.
\begin{figure}[H]
\centering
\generatedfigw{0.9}{p3-1}
\end{figure}
%
\item The below plot of \texttt{fa2} shows the frequency of the sound oscillate
twice. My ears agree.
\begin{figure}[H]
\centering
\generatedfigw{0.9}{p3-2}
\end{figure}
%
\end{enumerate}



\newpage
\chapter{Appendix: MATLAB Source Code}

What follows is a listing of the MATLAB source code (\autoref{lst:source-code}
and \autoref*{lst:mydftplot-m})---and the output of this code
(\autoref{lst:diary-txt})---used to generate the figures and other information
presented in this report.

\lstinputlisting[caption={The MATLAB script used for this report, \texttt{Lab06\_ahirzel.m}.},label={lst:source-code},style=MATLABcode]{Lab06_ahirzel.m}
\lstinputlisting[caption={The MATLAB script used to plot the DFTs shown in this report, \texttt{mydftplot.m}.},label={lst:mydftplot-m},style=MATLABcode]{mydftplot.m}
\lstinputlisting[caption={The output of listing \ref{lst:source-code}, \texttt{diary.txt}.},label={lst:diary-txt}]{generated/diary.txt}

\end{document}
