\documentclass{ajhlabreport}
\usepackage{multicol}
\usepackage{ajhdsp}

\newcommand{\generatedfigw}[2]{
	\includegraphics[width=#1\textwidth]{generated/#2.png}
}

\providecommand{\n}[0]{[n]}

\datedue{October 25, 2012}
\class{EE4252}
\submittedto{\href{mailto:yliu18@mtu.edu}{Yang Liu}}
\pretitle{Lab 5 Report}
\title{The DTFT}
\author{\href{mailto:ahirzel@mtu.edu}{Alex Hirzel}}

\begin{document}
\maketitle

%\chapter{Introduction}

%TODO



\chapter{Project 1(a): The Shift Property}

\pgfplotstableread{
3
3
3
2
2
2
1
1
1
}\dataone

\begin{figure}[H]
%
\subbottom[The signal $x\n = \smash{\{\marker{3}, 3, 3, 2, 2, 2, 1, 1, 1\}}$.\label{fig:p1-x}]{
\begin{tikzpicture}
\begin{axis}[discrete ir, half size, height=1in, xmax=8.25, ymax=3.4]
\addplot table[x expr=\coordindex, y index=0] \dataone;
\end{axis}
\end{tikzpicture}
}
%
\subbottom[The signal $y\n = \smash{\{3, 3, 3, 2, \marker{2}, 2, 1, 1, 1\}}$.\label{fig:p1-y}]{
\begin{tikzpicture}
\begin{axis}[discrete ir, half size, height=1in, xmin=-4.25, xmax=4.25, ymax=3.4, ylabel=$y\n$]
\addplot table[x expr=\coordindex-4, y index=0] \dataone;
\end{axis}
\end{tikzpicture}
}
%
\caption{Input signals $x[n]$ and $y[n]$ for Project 1. They are shifted
versions of one another related by $y[n] = x[n - 4]$.\label{fig:p1-inputs}}
\end{figure}

Comparing $x[n]$ and $y[n]$ from \autoref{fig:p1-inputs},

\newcommand{\inlinefigure}[1]{
\begin{figure}[H]
\centering
\generatedfigw{0.9}{#1}
\end{figure}
}

\renewcommand{\labelenumi}{(\alph{enumi})}
\begin{enumerate}
\item The magnitude spectra should be identical because the same frequencies are
present in each signal; the only distinction between the signals in this respect
is that each frequency component has peaks at different times; this amounts to a
phase shift. The identicality of the magnitude spectra is shown by the below
plot of $|X| - |Y|$, which has a $10^{-14}$ factor of scale on the $y$-axis.
%
\inlinefigure{p1a_magnitude}
%
\item The phase of the DFT of $x[n]$ will not be linear. This is due to the lack
of symmetry\footnote{i.e. $x[n]$ is not the shifted version of a signal that is
even or odd.} and is confirmed by the phase spectrum of $x[n]$ shown below.
%
\inlinefigure{p1a_phase}
%
\item The phase spectrum of $X$ does show jumps, but \texttt{unwrap} does not
remove them all. Jumps remain which are less than $2\pi$ as shown below.
%
\inlinefigure{p1a_phase_unwrapped}
%
\item The phase spectra for $X$ and $Y$ are not and should not be equal. They
should be related by a linear difference due to the group delay but will not be
equal.
\item Since the two signals are shifted version of one another in the discrete
domain, they will have a constant group delay and so the below unwrapped plot of
$\angle{X} - \angle{Y}$ is linear.
%
\inlinefigure{p1a_phase_diff}
%
\item From the plot above, delay can be found by calculating the slope. In this
case, the slope is $\approx 25$ and $25/(2\pi) \approx 3.979$. This matches the
discrete-domain delay which we know to be 4 samples.
\end{enumerate}



\chapter{Project 1(b): Periodicity, Convolution and Multiplication}

\begin{enumerate}
%
\item $|X(F)|$ should be (and is) periodic with $N = 1$. This is because $X(F)$
is generated from a discrete signal $x[n]$.
%
\inlinefigure{p1b_x}
%
\item $|X(F)|^2$ should be identical to $|G(F)|$ because $g[n] = x[n] \conv
x[n]$ and because convolution in one domain translates to multiplication in
another. The ratio of the below plots is 1.
%
\inlinefigure{p1b_xx_and_g}
%
\item $|Y_p(F)|$ and $|H(F)|$ are not identical (as shown below).
%
\inlinefigure{p1b_ypm_and_hm}
%
\item $|Y_p(F)|$ and $|H(F)|$ have a ratio of $\boxed{400}$.
%
\inlinefigure{p1b_ypm_over_hm}
%
\end{enumerate}



\newpage
\chapter{Project 1(c): Modulation}

\inlinefigure{p1c}

\begin{enumerate}
%
\item $|Y_1(F)|$ should (and does) consist of two half-height, shifted versions
of $X(F)$ (because the time-domain multiplication of a rect $x[t]$ and a
$\cos{(\cdots)}$ results in the convolution of a $\sinc{(\cdots)}$ and a pair of
impulses in the frequency domain).
%
\item $|Y_1(F)|$ and $|Y_2(F)|$ should be (and are) identical because
$\cos{(2.3n\pi)}$ aliases to $\cos{(0.3n\pi)}$ when $n$ is an integer.
%
\end{enumerate}



\chapter{Project 2: The DTFT of Digital Filters}
\begin{multicols}{3}
\renewcommand{\labelenumi}{\arabic{enumi}.}
\renewcommand{\labelenumii}{(\alph{enumii})}
\begin{enumerate}
%
\item \begin{enumerate} \item lowpass  \item IIR \item nonlinear-phase \end{enumerate}
%
\item \begin{enumerate} \item bandstop \item IIR \item nonlinear-phase \end{enumerate}
%
\item \begin{enumerate} \item highpass \item FIR \item linear-phase    \end{enumerate}
%
\columnbreak
%
\item \begin{enumerate} \item bandpass \item IIR \item nonlinear-phase \end{enumerate}
%
\item \begin{enumerate} \item lowpass  \item IIR \item linear-phase    \end{enumerate}
%
\item \begin{enumerate} \item lowpass  \item IIR \item linear-phase    \end{enumerate}
%
\columnbreak
%
\item \begin{enumerate} \item allpass  \item IIR \item nonlinear-phase \end{enumerate}
%
\item \begin{enumerate} \item comb     \item FIR \item linear-phase    \end{enumerate}
%
\newpage
%
\end{enumerate}
\end{multicols}



\newpage
\chapter{Appendix: MATLAB Source Code}

What follows is a listing of the MATLAB source code (listing
\ref{lst:source-code}) used to generate the figures and other information
presented in this report.

\lstinputlisting[caption={The MATLAB script used for this report, \texttt{Lab05\_ahirzel.m}.},label={lst:source-code},style=MATLABcode]{Lab05_ahirzel.m}

\end{document}
