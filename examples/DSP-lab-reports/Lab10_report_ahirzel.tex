\documentclass{ajhlabreport}
\usepackage{ajhdsp}
\usetikzlibrary{pgfplots.groupplots}

\newcommand{\generatedfigw}[2]{
	\includegraphics[width=#1\textwidth]{generated/#2.png}
}

\setlength{\parindent}{0bp}

\datedue{December 13, 2012}
\class{EE4252}
\submittedto{\href{mailto:yliu18@mtu.edu}{Yang Liu}}
\pretitle{Lab 10 Report}
\title{FIR Filters}
\author{\href{mailto:ahirzel@mtu.edu}{Alex Hirzel}}

\begin{document}
\maketitle




\chapter{Project 1: GUI-Based FIR Filter Design}%%%%%%%%%%%%%%%%%%%%%%%%%%%%%%%%

\begin{figure}[H]
\centering
\begin{tikzpicture}
\begin{groupplot}[discrete ir, height=2.75in, width=6.5in,
% http://tex.stackexchange.com/questions/36442/aligning-subplots-in-a-pgfplots-figure
yticklabel style={text width=2.35em,align=right},
% http://tex.stackexchange.com/questions/51095/reduce-font-size-and-keep-label-position-in-bar-chart-using-pgf-tikz
tick label style={font=\footnotesize},
ylabel=$H(f)$,
% http://tex.stackexchange.com/questions/54350/how-can-i-change-format-of-numbers-axis-on-pgfplots
every y tick label/.append style = { /pgf/number format/.cd, precision = 3, fixed },
group style = {
	group size = 1 by 3,
},
]
\nextgroupplot[xmax = 23.69, ymin = -0.3, ymax = 0.33, ytick = {-0.2349, -0.16614, 0.089025, 0.27548}]
\addplot table[x expr=\coordindex, y index=0] {generated/p1e2ir.csv};
\nextgroupplot[xmax = 24.72, ymin = -0.35, ymax = 0.45, ytick = {-0.2887, 0.079006, 0.39688}]
\addplot table[x expr=\coordindex, y index=0] {generated/p1e3ir.csv};
\nextgroupplot[xmax = 16.48, ymin = -0.35, ymax = 0.45, ytick = {-0.27043, 0.089186, 0.36413}]
\addplot table[x expr=\coordindex, y index=0] {generated/p1e4ir.csv};
\end{groupplot}
\end{tikzpicture}
\caption{Impulse responses for each example filter.}
\end{figure}

\begin{figure}[H]
\centering
\begin{tikzpicture}
\begin{axis}[continuous fr, height=3.5in, width=6.5in,
% http://tex.stackexchange.com/questions/36442/aligning-subplots-in-a-pgfplots-figure
yticklabel style={text width=2.35em,align=right},
% http://tex.stackexchange.com/questions/51095/reduce-font-size-and-keep-label-position-in-bar-chart-using-pgf-tikz
tick label style={font=\footnotesize},
ylabel=$-20 \log_{10} \abs{H(F)}$,
xmin = -0.53, xmax = 0.53,
ymin = -85, ymax = 5,
% http://tex.stackexchange.com/questions/54350/how-can-i-change-format-of-numbers-axis-on-pgfplots
every y tick label/.append style = { /pgf/number format/.cd, precision = 3, fixed },
legend pos = north east,
]
\addplot table[x expr=\coordindex/392 - 0.5, y index=0] {generated/p1e2mag.csv};
\addlegendentry{Example 2}
\addplot table[x expr=\coordindex/385 - 0.5, y index=0] {generated/p1e3mag.csv};
\addlegendentry{Example 3}
\addplot table[x expr=\coordindex/401 - 0.5, y index=0] {generated/p1e4mag.csv};
\addlegendentry{Example 4}
\end{axis}
\end{tikzpicture}
\caption{Bode plot of each filter.}
\end{figure}




\chapter{Project 2: GUI-Based Multi-Band Filter Design}%%%%%%%%%%%%%%%%%%%%%%%%%

\newcommand{\attempt}[3]{
\begin{figure}[H]
\centering
\begin{tikzpicture}
\begin{axis}[continuous fr, height=3.5in, width=6.5in,
% http://tex.stackexchange.com/questions/36442/aligning-subplots-in-a-pgfplots-figure
yticklabel style={text width=2.35em,align=right},
% http://tex.stackexchange.com/questions/51095/reduce-font-size-and-keep-label-position-in-bar-chart-using-pgf-tikz
tick label style={font=\footnotesize},
ylabel=$-20 \log_{10} \abs{H(F)}$,
xmin = 0, xmax = 0.53,
ymin = -85, ymax = 5,
% http://tex.stackexchange.com/questions/54350/how-can-i-change-format-of-numbers-axis-on-pgfplots
every y tick label/.append style = { /pgf/number format/.cd, precision = 3, fixed },
legend pos = north east,
]
\addplot table[x expr=\coordindex/1000, y index=0] {generated/p2attempt#1.csv};
\draw[dashed,gray,ultra thick] (axis cs:0,-40) -- (axis cs:0.5,-40);
#3
\end{axis}
\end{tikzpicture}
\caption{#2}
\end{figure}
}

\attempt{1}{Bode plot of first attempt at designed filter. Note that it
obviously does not meet stopband specs. To rememdy this, I will first try making
passband 2 order 39.}{}

\attempt{2}{Bode plot of second attempt at designed filter. There is one
problematic sidelobe which I will mitigate by changing $A_\text{s} = 40\deci\bel
+ 1.131\deci\bel = 41.131\deci\bel$.}{
\node[coordinate,pin=above right:{\color{red}Peak: $-38.69\deci\bel$}] at (axis cs:0.363,-38.686) {};
}

\attempt{3}{Third time's the charm! This filter meets specs.}{}


\newpage
\chapter{Appendix: MATLAB Source Code}

What follows is a listing of the MATLAB source code (\autoref{lst:source-code})
and the text output of this code (\autoref{lst:diary-txt}) used to generate the
figures and other information presented in this report.

\lstinputlisting[caption={The MATLAB script used for this report,
\texttt{Lab10\_ahirzel.m}.},label={lst:source-code},style=MATLABcode]{Lab10_ahirzel.m}
\lstinputlisting[caption={The output of listing \ref{lst:source-code}, \texttt{diary.txt}.},label={lst:diary-txt}]{generated/diary.txt}

\end{document}
