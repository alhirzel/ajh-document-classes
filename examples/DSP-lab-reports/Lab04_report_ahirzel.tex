\documentclass{ajhlabreport}
\usepackage{ajhdsp}
\usepackage{verbatim}

\newcommand{\generatedfigw}[2]{
	\includegraphics[width=#1\textwidth]{generated/#2.png}
}

\providecommand{\n}[0]{[n]}

\datedue{October 18, 2012}
\class{EE4252}
\submittedto{\href{mailto:yliu18@mtu.edu}{Yang Liu}}
\pretitle{Lab 4 Report}
\title{The $z$-transform and its applications}
\author{\href{mailto:ahirzel@mtu.edu}{Alex Hirzel}}

\begin{document}
\maketitle



\chapter{Project 1}

The results of \texttt{ztr} and \texttt{izt} matched expectations; this can be
confirmed by referring to \autoref{lst:diary-txt} and noting the final symbolic
representations are equivalent to the inputs. That is to say,
\[ \verb|(5*(2 .^n)).*ustep(n)|             \equiv 5 (2)^n u(n) \]
and
\[ \verb|(5*(2 .^n)-3*((-1).^n)).*ustep(n)| \equiv 5 (2)^n u(n) - 3 (-1)^n u(n)\text{.} \]

\begin{figure}[H]
\centering
\subbottom[Total response of system A.\label{fig:p1-a-total}]{\generatedfigw{0.45}{p1_2a_total}}
\subbottom[Impulse response of system A.\label{fig:p1-a-ir}]{\generatedfigw{0.45}{p1_2a_ir}}
\subbottom[Total response of system B.\label{fig:p1-b-total}]{\generatedfigw{0.45}{p1_2b_total}}
\subbottom[Impulse response of system B.\label{fig:p1-b-ir}]{\generatedfigw{0.45}{p1_2b_ir}}
\caption{The outputs from convolution in Project 1.\label{fig:p1}}
\end{figure}

The estimated steady-state value for system A is 0.6.



\newpage
\chapter{Project 2}

An FIR filter was created to attempt to remove the $60\hertz$ noise from the ECG
signal. Zeroes were placed at the locations in the $z$-plane corresponding to the
$60\hertz$ component when sampled at $300\hertz$. That is to say, conjugate
zeroes were placed at $F = \pm 60/300=0.2$ on the unit circle, resulting in
\begin{align}
%
H_\text{FIR}(z) = \frac{1}{(z-e^{j 2\pi 60/300})(z-e^{-j 2\pi 60/300})}
\text.
\label{eq:p2-fir}
%
\end{align} The output of this filter is shown in \autoref{fig:p2-fir}. A sister
filter was designed by placing zeroes very near to these poles to reduce
overshoot. This is an IIR filter because the representation of $h[n]$ (not
shown) has only an infinite representation. \begin{align}
%
H_\text{IIR}(z) = \frac{(z - 0.99 e^{j 2\pi 60/300})(z - 0.99 e^{-j 2\pi 60/300})}{(z - e^{j 2\pi 60/300})(z - e^{-j 2\pi 60/300})}
\text.
\label{eq:p2-iir}
%
\end{align} The output of this IIR filter is shown in \autoref{fig:p2-iir}. Note
that there is a \emph{ringing} present at start-up which damps out over less
than a quarter period. This ringing can be traded for overshoot by manipulating
the radial distance of the poles\footnote{that is to say, the closer the poles
are to the unit circle, the less overshoot and more ringing is present; when the
poles are moved further from the unit circle, ringing is reduced but overshoot
is increased.}. Finally, an averaging filter was developed with a width of
$360/60=5$ samples in hopes that it would eliminate the sinusoidal corruption.
The output from this filter is shown in \autoref{fig:p2-ma}.

\begin{figure}[H]
\centering
\subbottom[Pole-zero plot of \autoref{eq:p2-fir}.\label{fig:p2-fir-pz}]{
%
\begin{tikzpicture}
\begin{axis}[pz plot]
\addplot[data cs=polar,mark=o] coordinates { (60/300*180,1) (-60/300*180,1) };
\end{axis}
\end{tikzpicture}
%
}
\subbottom[Pole-zero plot of \autoref{eq:p2-iir}.\label{fig:p2-iir-pz}]{
%
\begin{tikzpicture}
\begin{axis}[pz plot]
\addplot[data cs=polar,mark=o] coordinates { (60/300*180,1)   (-60/300*180,1) };
\addplot[data cs=polar,mark=x] coordinates { (60/300*180,0.9) (-60/300*180,0.9) };
\end{axis}
\end{tikzpicture}
%
}
\caption{Pole-zero plots of the FIR and IIR filters.\label{fig:p2-pz}}
\end{figure}


\subsection{Conclusions}

Of all approaches to filtering the $60\hertz$ noise, it must be said that the
moving average was all-round most effective except for the phase lag of 2.5
samples. If the end application of this filter cannot tolerate this distortion
the next best choice is the IIR filter assuming the visible ringing on start-up
can be ignored. (Refer to \autoref{fig:p2-iir} on the left-hand side to view the
ringing.)

\begin{figure}[p]
\centering
\subbottom[Filtering the ECG signal with an FIR filter, \autoref{eq:p2-fir}.\label{fig:p2-fir}]{\generatedfigw{1}{p2_fir}}
\subbottom[Filtering the ECG signal with an IIR filter, \autoref{eq:p2-iir}.\label{fig:p2-iir}]{\generatedfigw{1}{p2_iir}}
\subbottom[Filtering the ECG signal with a traditional moving average filter with 5 samples.\label{fig:p2-ma}]{\generatedfigw{1}{p2_ma}}
\caption{Various filtering techniques applied to an ECG signal contaminated
with $60\hertz$ noise.\label{fig:p2}}
\end{figure}



\newpage
\chapter{Project 3}

Note that since $S=8192\hertz$, $\tau$ is equivalent to 768 samples (and
$2\tau$ to 1536 samples), so in the $z$-domain, \begin{align*}
%
c(t) &= s(t) + \alpha_1 s(t - \tau) + \alpha_2 s(t - 2\tau) \\
%
\therefore\ 
C(z) &= S(z) + 0.9z^{-768} S(z) + 0.8z^{-1536} S(z)
%
\end{align*} and so the transfer function of the system can be found and inverted \begin{align*}
%
H(z)          &= \frac{C(z)}{S(z)} = \frac{1}{1 + 0.9z^{-768} + 0.8z^{-1536}}\text{,} \\
%
H_\text{I}(z) &= [H(z)]^{-1} = 1 + 0.9z^{-768} + 0.8z^{-1536}\text{.}
%
\end{align*} Applying this filter to the corrupted signal results in a
computerized voice stating \[\boxed{\text{\textbf{"DSP is fun"},}}\] a statement with
which this author generally agrees.



\newpage
\chapter{Appendix: MATLAB Source Code}

What follows is a listing of the MATLAB source code (listing
\ref{lst:source-code})---and the output of this code (listing
\ref{lst:diary-txt})---used to generate the figures and other information
presented in this report.

\lstinputlisting[caption={The MATLAB script used for this report, \texttt{Lab04\_ahirzel.m}.},label={lst:source-code},style=MATLABcode]{Lab04_ahirzel.m}
\lstinputlisting[caption={The output of listing \ref{lst:source-code}, \texttt{diary.txt}.},label={lst:diary-txt}]{generated/diary.txt}

\end{document}
